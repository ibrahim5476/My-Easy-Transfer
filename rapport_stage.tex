% Rapport de stage PFE
% École Nationale des Sciences et Technologies Avancées Borj Cedria
% Projet : my_easy_transfer
% Auteur : Majdoue Brams
% Année universitaire : 2023-2024

\documentclass[12pt,a4paper]{report}
\usepackage[utf8]{inputenc}
\usepackage[T1]{fontenc}
\usepackage[french]{babel}
\usepackage{graphicx}
\usepackage{geometry}
\usepackage{hyperref}
\usepackage{lipsum}
\usepackage{listings}
\usepackage{color}
\usepackage{float}
\usepackage{array}
\usepackage{longtable}
\usepackage{tikz}
\usepackage{pgf-umlcd}
\usepackage{enumitem}
\usepackage{fancyhdr}
\usepackage{booktabs}
\usepackage{setspace}
\onehalfspacing

% Configuration des listings pour le code
\definecolor{codegreen}{rgb}{0,0.6,0}
\definecolor{codegray}{rgb}{0.5,0.5,0.5}
\definecolor{codepurple}{rgb}{0.58,0,0.82}
\definecolor{backcolour}{rgb}{0.95,0.95,0.92}

\lstdefinestyle{mystyle}{
    backgroundcolor=\color{backcolour},   
    commentstyle=\color{codegreen},
    keywordstyle=\color{magenta},
    numberstyle=\tiny\color{codegray},
    stringstyle=\color{codepurple},
    basicstyle=\ttfamily\footnotesize,
    breakatwhitespace=false,         
    breaklines=true,                 
    captionpos=b,                    
    keepspaces=true,                 
    numbers=left,                    
    numbersep=5pt,                  
    showspaces=false,                
    showstringspaces=false,
    showtabs=false,                  
    tabsize=2
}
\lstset{style=mystyle}

% Configuration de la géométrie
\geometry{left=2.5cm,right=2.5cm,top=2.5cm,bottom=2.5cm}

% Configuration des en-têtes
\pagestyle{fancy}
\fancyhf{}
\fancyhead[L]{PFE ENSTAB 2023-2024}
\fancyhead[R]{\thepage}
\fancyfoot[C]{Majdoue Brams - my_easy_transfer}

\title{Rapport de Stage\\Développement d'un agent virtuel pour le transfert d'argent et la recharge mobile après vérification faciale et vocale}
\author{Majdoue Brams}
\date{\today}

\begin{document}

\maketitle

\tableofcontents

% Page de garde
\begin{titlepage}
    \begin{center}
        \includegraphics[width=0.3\textwidth]{logo_enstab.png}~\\[1cm]
        {\LARGE \textbf{École Nationale des Sciences et Technologies Avancées Borj Cedria}}\\[0.5cm]
        {\large Département Informatique}\\[2cm]
        {\Huge \textbf{Rapport de Projet de Fin d'Études}}\\[1cm]
        {\Large Sujet : \textbf{my_easy_transfer}}\\[0.5cm]
        {\large Plateforme Sécurisée de Transfert d'Argent avec Vérification Biométrique}\\[2cm]
        \textbf{Présenté par :}\\
        Majdoue Brams\\[0.5cm]
        \textbf{Encadrant académique :}\\
        Dr. Ahmed Ben Salah\\[0.5cm]
        \textbf{Encadrant entreprise :}\\
        Mme. Leila Trabelsi\\[2cm]
        Année universitaire : 2023-2024
    \end{center}
\end{titlepage}

% Dédicaces
\chapter*{Dédicaces}
\addcontentsline{toc}{chapter}{Dédicaces}
À mes parents, pour leur soutien inconditionnel,\\
À mes professeurs, pour leur dévouement,\\
À tous ceux qui m'ont accompagné dans ce parcours.\\
\newpage

% Remerciements
\chapter*{Remerciements}
\addcontentsline{toc}{chapter}{Remerciements}
\lipsum[1-2]

% Résumé
\chapter*{Résumé}
\addcontentsline{toc}{chapter}{Résumé}
\lipsum[3-4]

\textbf{Mots-clés :} Django, Biométrie, Intelligence Artificielle, Sécurité, Transactions Financières

% Abstract
\chapter*{Abstract}
\addcontentsline{toc}{chapter}{Abstract}
\lipsum[5-6]

\textbf{Keywords:} Django, Biometrics, Artificial Intelligence, Security, Financial Transactions

% Table des matières
\listoffigures
\listoftables
\newpage

% Introduction générale
\chapter*{Introduction}
\addcontentsline{toc}{chapter}{Introduction}
Ce rapport présente le travail réalisé lors de mon stage portant sur le développement d'un agent virtuel permettant le transfert d'argent et la recharge mobile, intégrant des mécanismes de vérification faciale et vocale pour renforcer la sécurité des transactions. Ce document détaille le cadre du stage, les technologies utilisées, ainsi que les différentes étapes de la réalisation du projet.

% CHAPITRE 1 : Cadre de stage
\chapter{Cadre de stage}
\section{Présentation de l'entreprise}
% -- À compléter

\section{Objectifs du stage}
% -- À compléter

\section{Contexte du projet}
% -- À compléter ou reprendre le texte existant

\section{Problématique}
% -- À compléter ou reprendre le texte existant

% CHAPITRE 2 : Technologies utilisées
\chapter{Technologies utilisées}
\section{Langages de programmation}
% -- À compléter (ex: Python, JavaScript, etc.)

\section{Frameworks et bibliothèques}
% -- À compléter (ex: TensorFlow, OpenCV, etc.)

\section{Outils de développement}
% -- À compléter (ex: Git, Docker, etc.)

% CHAPITRE 3 : Réalisation
\chapter{Réalisation}
\section{Analyse des besoins}
% -- À compléter

\section{Architecture du système}
% -- À compléter

\section{Développement de la vérification faciale}
% -- À compléter

\section{Développement de la vérification vocale}
% -- À compléter

\section{Intégration et tests}
% -- À compléter

% Conclusion
\chapter*{Conclusion}
\addcontentsline{toc}{chapter}{Conclusion}
% -- À compléter

% Bibliographie
\chapter*{Bibliographie}
\addcontentsline{toc}{chapter}{Bibliographie}
\begin{thebibliography}{9}
\bibitem{django} Django Documentation, \url{https://docs.djangoproject.com/}
\bibitem{face} face-recognition Documentation, \url{https://github.com/ageitgey/face_recognition}
\bibitem{openai} OpenAI API Documentation, \url{https://platform.openai.com/docs/}
\bibitem{security} OWASP Web Security Testing Guide
\bibitem{biometrics} Handbook of Biometric Anti-Spoofing
\bibitem{python} Python Documentation, \url{https://docs.python.org/3/}
\bibitem{ml} Machine Learning for Computer Vision
\end{thebibliography}

\end{document}